% $Author: oscar $
% $Date: 2009-11-06 14:37:12 +0100 (Fri, 06 Nov 2009) $
% $Revision: 29604 $
%=============================================================
% Demo the st80 listings package
%=============================================================
\documentclass[11pt,a4paper]{article}
\input{st80.tex}
%=============================================================
\begin{document}
%=============================================================
\section*{Code environments using the listings package}
%=============================================================

\begin{verbatim}
\begin{code}
just some plain code
\end{code}
\end{verbatim}
\begin{code}
just some plain code
\end{code}

%=============================================================
\subsection*{Listings environments and macros}
The code environments

\begin{verbatim}
\begin{code}
...
\end{code}
\end{verbatim}
take plain, verbatim code,
and translate some special characters like $\wedge$ to \ct{^}. Even tabs are handled, (which is not true for verbatim).
\begin{code}
"All handled correctly: ^ $ ' % \\ << >> _ { }"
"NB: If you !{\bf really}! want an exclamation mark you must spell it BANG"
| y |
true & false not & (nil isNil) ifFalse: [self halt].
y _ self size + super size.
#($a #a 'a' 1 1.0)
	do: [:each | Transcript
			show: (each class name);
                     show: ' ';
                     show: (each printString).
{ 1 + 2 . 3 \\ 4 . 1 << 3. 2 >> 5 . 1 % 2 }.
^ x < y
\end{code}

QWERTY layout:
\begin{code}
BANG @ # $ % ^ & * ( ) UNDERSCORE +
1 2 3 4 5 6 7 8 9 0 - =
 Q W E R T Y U I O P { }
 q w e r t y u i o p [ ]
  A S D F G H J K L : "" |   		 (twice " to turn off italics")
  a s d f g h j k l ; ' \
   Z X C V B N M < > ?
   z x c v b n m , . /
\end{code}

LaTeX escape:
\begin{verbatim}
\begin{code}
plain code and !\textbf{bolded text}!
\end{code}
\end{verbatim}

\begin{code}
plain code and !\textbf{bolded text}!
\end{code}

% ^ $ \\ % # '

In-line code with \verb|\ct| is typed like this \verb|\ct{1 + 2 --> 3}| and looks like this: \ct{1 + 2 --> 3}, text can follow immediately.  The ``brackets'' around \verb|\ct| can be any matching pair of characters, useful if you want \ct${ and }$ in the code.

%=============================================================
\subsection*{Special chars with $\backslash$ct}
\ct@^ ~ # $ ' % \\ << >> _ {  } ! -- --> @\\
\verb|\ct@^ ~ # $ ' % \\ << >> _ {  } ! -- --> @|

%=============================================================
\subsection*{Special conventions}

\verb$\ct{Class>>>method}$ prints as \ct{Class>>>method}.\\
\verb$\ct{3 + 4 - 5 --> 2}$ prints as \ct{3 + 4 - 5 --> 2}.
%=============================================================
\end{document}
%=============================================================
