% $Author$
% $Date$
% $Revision$
%=================================================================
\ifx\wholebook\relax\else
% --------------------------------------------
% Lulu:
	\documentclass[a4paper,10pt,twoside]{book}
	\usepackage[
		papersize={6.13in,9.21in},
		hmargin={.815in,.815in},
		vmargin={.98in,.98in},
		ignoreheadfoot
	]{geometry}
	\input{../common.tex}
	\pagestyle{headings}
	\setboolean{lulu}{true}
% --------------------------------------------
% A4:
%	\documentclass[a4paper,11pt,twoside]{book}
%	\input{../common.tex}
%	\usepackage{a4wide}
% --------------------------------------------
    \graphicspath{{figures/} {../figures/}}
	\begin{document}
	\renewcommand{\nnbb}[2]{} % Disable editorial comments
	\sloppy
\fi
%=================================================================
\chapter{Frequently Asked Questions}
\label{app:faq}

\on{These should be *real* (not invented) FAQs.
Here are a few that I have collected.
Check the ST lecture notes for more FAQs.
We should also try to mine more from newbies.}

%=================================================================
\section{Getting started}
\begin{faq}
Where do I get the latest Squeak?
\end{faq}
\answer
\url{https://squeak.org/downloads/}
\index{download}

\begin{faq}
Which \sq image should I use with this book?
\end{faq}
\answer
We recommend that you use the standard image provided on \url{https://squeak.org/downloads/}.
You should also be able to use most other images, but you may find that the hands-on exercises behave differently in surprising ways.

\begin{faq}
I have destroyed my world docking bar at the top of the world in some way.
How do I get it back?
\end{faq}
\answer
Open the yellow button menu of the world (by clicking the yellow button in the world). 
There you find the menu item \menu{show main docking bar}.
Click it once to deactivate it completely, and then click it a second time to re-activate it.

\begin{faq}
How do I install more than what is already in the image?
\end{faq}
\answer
There are several ways to install additional packages. 
First, you can always download sar- or mcz-files and simply drop them into \sq.
Second, you can use the class \clsind{Installer}, which provides methods for installing several kinds of packages from different sources, \eg Squeaksource. 
To start using the \ct{Installer} you can take a look at the class comment.
Finally, you can currently find an increasing number of \sq projects on GitHub, many of them to be installed via \ind{Metacello}.
Metacello is a configuration management project for specifying dependencies between \st projects.
Before you can install a project via Metacello you may have to setup Metacello in your environment. 
You can do so via \ct{Installer ensureRecentMetacello}.

\begin{faq}
Some code triggered a lot of debuggers and my whole screen is filled with them. 
How can I get rid of them?
\end{faq}
\answer
You can close all windows that look similar at once by selecting \menu{World docking bar \go Windows \go Bad Window \go Close all like this}.

%=================================================================
\section{Collections}

\begin{faq}
How do I sort a \clsind{Collection}?
\end{faq}
\answer
If you want a sorted copy, you can use \mthind{Collection}{sorted}. 
If you want to have a collection that keeps the elements sorted, even when you add or remove elements, send the message \mthind{Collection}{asSortedCollection}.
Finally, some collections can be sorted in-place as they understand \ct{sort}, which will not return a copy, but re-order the collection.

\begin{code}{@TEST}
collection := {7 . 2 . 6 . 1}.
collection sorted. --> #(1 2 6 7)
collection sorted == collection. --> false

collection asSortedCollection. --> a SortedCollection(1 2 6 7)

collection sort. --> #(1 2 6 7)
collection. --> #(1 2 6 7)
collection sort == collection. --> true
\end{code}

To define custom sort criteria, use a sort block or a function (see \secref{sec:sorting}):

\begin{code}{@TEST}
collection := #('Foo' 'Bar' 'baz').
collection sorted. --> #('Bar' 'Foo' 'baz')

collection sorted: [:a :b | a caseInsensitiveLessOrEqual: b]. --> #('Bar' 'baz' 'Foo')
collection sorted: [:a | a asUppercase] ascending. --> #('Bar' 'baz' 'Foo')
collection sorted: [:a | a asUppercase] descending. --> #('Foo' 'baz' 'Bar')

Fix(#(1 0 2) @ #(2 0 1)) sorted: #r ascending , #x descending. --> {0@0 . 2@1 . 1@2}
\end{code}

\begin{faq}
How do I convert a collection of characters to a \clsind{String}?
\end{faq}
\answer
\begin{code}{@TEST}
collection := 'hello' asSet.
String newFrom: collection asArray. --> 'oelh'
\end{code}

%=================================================================
\section{Browsing the system}

\begin{faq}
How do I search for a class?
\end{faq}
\answer
\short{b} (browse) on the class name, \short{f} in the category pane of the class browser, or the global search on the top right of the world.
\index{keyboard shortcut!browse it}
\index{keyboard shortcut!find ...}

\begin{faq}
How do I browse all references to a given class?
\end{faq}
\answer
Select the class name and press \short{N}, or open the yellow button menu and select \menu{more \go references to it}.

\begin{faq}
How do I find/browse all sends to super?
\end{faq}
\answer
The second solution is much faster:
\begin{code}{}
SystemNavigation default browseMethodsWithSourceString: 'super'.
SystemNavigation default browseAllSelect: [:method | method sendsToSuper ].
\end{code}
\index{browsing programmatically}
\clsindex{SystemNavigation}

\begin{faq}
How do I browse all super \subind{super}{send}{}s within a hierarchy?
\end{faq}
\answer
\begin{code}{}
browseSuperSends := [:aClass | SystemNavigation default
	browseMessageList: (aClass withAllSubclasses gather: [ :each |
		(each methodDict associations
			select: [ :assoc | assoc value sendsToSuper ])
				collect: [ :assoc | MethodReference class: each selector: assoc key ] ])
	name: 'Supersends of ' , aClass name , ' and its subclasses'].
browseSuperSends value: OrderedCollection.
\end{code}

\begin{faq}
How do I find out which new methods are introduced by a class?
(\ie, not including overridden methods.)
\end{faq}
\answer
Here we ask which methods are introduced by \ct{TranslucentColor}:
\begin{code}{@TEST | newMethods |}
newMethods := [:aClass| aClass methodDict keys select:
	[:aMethod | (aClass superclass canUnderstand: aMethod) not ]].
newMethods value: TranslucentColor --> #(#setRgb:alpha:)
\end{code}

\begin{faq}
How do I tell which methods of a class are abstract?
\end{faq}
\answer
\begin{code}{@TEST | abstractMethods |}
abstractMethods :=
	[:aClass | aClass methodDict keys select:
		[:aMethod | (aClass>>aMethod) isAbstract ]].
abstractMethods value: Collection --> #(#add: #remove:ifAbsent: #do:)
\end{code}

\begin{faq}
How do I generate a view of the \ind{AST} of an expression?
\end{faq}
\answer
Load AST from squeaksource.com. Then evaluate:
\begin{code}{}
(RBParser parseExpression: '3 + 4') explore
\end{code}
Alternatively you can also directly \emph{explore it}.
\clsindex{RBParser}

\begin{faq}
How do I find all the Traits in the system?
\end{faq}
\answer
\begin{code}{}
Smalltalk allTraits
\end{code}

\begin{faq}
How do I find which classes use traits?
\end{faq}
\answer
\begin{code}{}
Smalltalk allClasses select: [:each | each hasTraitComposition]
\end{code}

%=================================================================
\section{Morphic}
\begin{faq}
How can I show images in Morphic?
\end{faq}
\answer
An first and easy way to get an image into \sq is by dragging the image file into the environment and dropping it directly within the world and selecting \menu{open graphic in a window}.
This will create an \clsind{ImageMorph} in the world.

\noindent Alternatively you can also create an \ct{ImageMorph} through code, for example using the following expression, if the image file is in the default folder:
\begin{code}{}
ImageMorph new
    image: (Form fromFileNamed: 'anImage.jpg');
    openInHand.
\end{code}  

\begin{faq}
How do I center text in a \clsind{TextMorph}?
\end{faq}
\answer
The following expression will do this:
\begin{code}{}
TextMorph new
    contents: 'Hello';
    hResizing: #rigid; "States that the morph should not change 
                        its width to match the width of the content"
    width: 300;
    centered;
    openInWorld.
\end{code}  

\begin{faq}
How do I add formatting to text?
\end{faq}
\answer
The relevant classes are \clsind{Text} and \clsind{TextAttribute}. 
You can add \ct{TextAttribute}s to section of text by using the messages in the \ct{emphasis} protocol of \ct{Text}, for example:
\begin{code}{}
text := 'yellow and green' asText.
text 
	addAttribute: TextColor yellow from: 1 to: 6;
	addAttribute: TextEmphasis italic from: 1 to: 6;
	addAttribute: TextColor green from: 12 to: 16;
	addAttribute: TextEmphasis bold from: 12 to: 16.
text asMorph openInHand.
\end{code}  

%=================================================================
\section{System}

\begin{faq}
How can I achieve concurrency?
\end{faq}
\answer
There are a number of classes to create and control \ind{concurrency}.
You can find most of them, such as \clsind{Process}, \clsind{Mutex}, or \clsind{Promise} in the system category \ct{Kernel-Processes}.
Starting new processes is done via the messages in the \ct{scheduling} protocol of \clsind{BlockClosure}.
Note, that, like most interpreter-based languages to date, the \sq environment runs in a single operating system process, so you will not benefit from parallel hardware.
Writing concurrent code might still be useful for background processing of long computations or I/O.

\begin{faq}
I have created concurrent behavior, but lost track of some of the processes I started. 
What can I do?
\end{faq}
\answer
There is a tool called \ind{Process Browser} which you can either open from \menu{World menu \go open} or \menu{World docking bar \go Tools}. 
The process browser lists all processes and the yellow button menu allows you to control the process, \eg terminate it.

\begin{faq}
How do I express OS-independent file paths?
\end{faq}
\answer
A very manual way to achieve this is to construct paths using OS-specific path delimiters by using \mthind{FileDirectory}{slash}.
\ct{FileDirectory} instances can be created with any kind of path by sending the message \ct{uri:}.
Finally, if you already have a \ct{FileDirectory} instance you can send the message \ct{\} to navigate it, for example:
\begin{code}{}
(FileDirectory default / 'release-notes' / '5.0') readStream contents lines first.
\end{code}

%=================================================================
\section{Using Monticello and SqueakSource}

\begin{faq}
How do I load a \ind{Squeaksource} project?
\index{Monticello}
\end{faq}
\answer
\begin{enumerate}
  \item Find the project you want in \url{squeaksource.com}
  \item Copy the registration code snippet
  \item Select \menu{World docking bar \go Tools \go Monticello browser}
  \item Select \menu{+Repository \go HTTP}
  \item Paste and accept the Registration code snippet; enter your password
  \item Select the new repository and \menu{Open} it
  \item Select and load the latest version
\end{enumerate}

\begin{faq}
How do I create a SqueakSource project?
\end{faq}
\answer
\begin{enumerate}
  \item Go to \url{squeaksource.com}
  \item Register yourself as a new member
  \item Register a project (name = category)
  \item Copy the Registration code snippet
  \item \menu{open \go Monticello browser}
  \item \menu{+Package} to add the category
  \item Select the package
  \item \menu{+Repository \go HTTP}
  \item Paste and accept the Registration code snippet; enter your password
  \item \menu{Save} to save the first version
\end{enumerate}

\begin{faq}
How do I extend \ct{Number} with \ct{Number>>>chf} but have Monticello recognize it as being part of my \ct{Money} project?
\end{faq}
\answer
Put it in a method-category named \ct{*Money}.
Monticello gathers all methods that are in other categories named like *package and includes them in your package.

%=================================================================
\section{Tools}

\begin{faq}
Doing everything with the mouse take so much time. 
Which keyboard shortcuts are there?
\end{faq}
\answer
You can find a list of available keyboard shortcuts directly in the environment at \menu{World docking bar \go Help \go Keyboard Shortcuts}.

\begin{faq}
Some menus are quite large and I struggle to find what I am looking for.
\end{faq}
\answer
Menus, like all panes, can be filtered by simply typing on your keyboard. 
If you type the complete string of the menu item you can also directly activate it by pressing return.

\begin{faq}
Some message sends seem slow to me.
How do I find out what makes them slow?
\end{faq}
\answer
You can use the \ct{MessageTally} tool that allows you to profile your code. 
To profile all processes currently running in the process, you can open the tool from \menu{World docking bar \go Extras \go Start Profiler}\index{Profiler}.
If you only want to profile a specific message send, you can use \mthind{MessageTally class}{spyOn:}, for example:
\begin{code}{}
MessageTally spyOn: [10000 timesRepeat: [1.23 printString]]
\end{code}
You can achieve the same by selecting an expression and select \menu{spy on it} from the yellow button menu.

\begin{faq}
How do I programmatically open the \ind{SUnit} \ind{TestRunner}?
\end{faq}
\answer
Evaluate \ct{TestRunner open}.

\begin{faq}
Where can I find the \ind{Refactoring Browser}?
\end{faq}
\answer
Load AST then Refactoring Engine from squeaksource.com:
\url{www.squeaksource.com/AST}
\url{www.squeaksource.com/RefactoringEngine}

\begin{faq}
How do I register the browser that I want to be the default?
\end{faq}
\answer
Click the menu icon in the top left of the Browser window.

%=================================================================
\section{Regular expressions and parsing}

\begin{faq}
Where is the documentation for the RegEx package?
\end{faq}
\answer
Look at the \menu{DOCUMENTATION} protocol of \ct{RxParser class} in the \menu{Regex-Core} category.

\begin{faq}
Are there tools for writing parsers?
\end{faq}
\answer
You can use \ind{SmaCC}\,---\,the Smalltalk Compiler Compiler (load it from \url{https://www.squeaksource.com/SmaccDevelopment.html}).
Or you can use \ind{Ohm/S}, a \st implementation of Ohm/JS (load it from \url{https://www.github.com/hpi-swa/Ohm-S}).

%=================================================================
\ifx\wholebook\relax\else\end{document}\fi
%=================================================================

