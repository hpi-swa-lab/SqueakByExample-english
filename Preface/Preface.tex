% $Author$
% $Date$
% $Revision$
%=================================================================
\ifx\wholebook\relax\else
% --------------------------------------------
% Lulu:
	\documentclass[a4paper,10pt,twoside]{book}
	\usepackage[
		papersize={6.13in,9.21in},
		hmargin={.815in,.815in},
		vmargin={.98in,.98in},
		ignoreheadfoot
	]{geometry}
	\input{../common.tex}
	\pagestyle{headings}
	\setboolean{lulu}{true}
% --------------------------------------------
% A4:
%	\documentclass[a4paper,11pt,twoside]{book}
%	\input{../common.tex}
%	\usepackage{a4wide}
% --------------------------------------------
    \graphicspath{{figures/} {../figures/}}
	\begin{document}
	\renewcommand{\nnbb}[2]{} % Disable editorial comments
	\frontmatter
\fi
%=================================================================
\chapter{Preface}\label{cha:intro}

%=================================================================
\section*{What is \sq?}

\sq is a modern, open-source, fully-featured implementation of the \st programming language and environment.

\sq is highly portable --- even its virtual machine is written almost entirely in \st, making it easy to debug, analyze, and change.
\sq is the vehicle for a wide range of innovative projects from multimedia applications and educational platforms to commercial web development environments.

%=================================================================
\section*{Who should read this book?}

This book presents the various aspects of \sq, starting with the basics, and proceeding to more advanced topics.

This book will not teach you how to program.
The reader should have some familiarity with programming languages.
Some background on object-oriented programming is also helpful.

This book will introduce the \sq programming environment, the language, and the associated tools.
You will be exposed to common idioms and practices, but the focus is on the technology, not on object-oriented design.
Wherever possible, we will show you lots of examples.
(We have been inspired by Alec Sharp's excellent book on Smalltalk~\cite{Shar97a}.)
\index{Sharp, Alex}

\ifluluelse{}{\newpage} % layout hint
%=================================================================
\section*{A word of advice}

% http://www.surfscranton.com/architecture/KnightsPrinciples.htm

Do not be frustrated by parts of \st that you do not immediately understand.
You do not have to know everything!
Alan Knight expresses this principle as follows\footnote{\url{https://web.archive.org/web/20191012144419/http://alanknightsblog.blogspot.com/2011/10/principles-of-oo-design-or-everything-i.html}}:
\index{Knight, Alan}
\important{{\bf Try not to care.}
	Beginning \st programmers often have trouble because they think they need to understand all the details of how a thing works before they can use it.
	This means it takes quite a while before they can master \ct{Transcript show: 'Hello World'}.
	One of the great leaps in OO is to be able to answer the question ``How does this work?'' with ``I don't care''.%
}

%=================================================================
\section*{An open book}

This book is an open book in the following senses:

\begin{itemize}

\item	The content of this book is released under the Creative Commons Attribution-ShareAlike (by-sa) license.
		In short, you are allowed to freely share and adapt this book, as long as you respect the conditions of the license available at the following URL:
		\url{creativecommons.org/licenses/by-sa/3.0/}.

\item	This book just describes the core of \sq.
		Ideally, we would like to encourage others to contribute chapters on the parts of \sq that we have not described.
		If you would like to participate in this effort, take a look at the repository at \sbeRepoUrl.
		We would like to see this book grow!
\end{itemize}

%=================================================================
\section*{The \sq community}

The \sq community is friendly and active.
Here is a short list of resources that you may find useful:

\begin{itemize}
\item \url{www.squeak.org} is the main web site of \sq.
(Do not confuse it with \url{www.squeakland.org} which is dedicated to the eToys environment built on top of \sq but whose audience is elementary school teachers.)

\item \url{www.squeaksource.com} is the equivalent of GitHub for \sq projects.

\item \url{wiki.squeak.org/squeak} is a wiki with all kinds of information about \sq.
\end{itemize}

\paragraph{About mailing-lists.}
There are a lot of mailing-lists and sometimes they can be just a little bit too active.
If you do not want to get flooded by mail but would still like to participate we suggest you use \url{forum.world.st} or \url{forum.world.st/Squeak-f45487.html} to browse the lists.

You can find the complete list of \sq mailing-lists at \url{lists.squeakfoundation.org/mailman/listinfo}.

\begin{itemize}
\item Note that \emph{squeak-dev} refers to the developers' mailing list, which can be browsed here:\\
\url{forum.world.st/squeak-dev-f45488.html}
\item \emph{Beginners} refers to a friendly mailing-list for beginners where any question can be asked:\\
\url{forum.world.st/Squeak-Beginners-f107673.html}\\
(There is so much to learn that we are all beginners in some aspect of \sq!)
\end{itemize}

\paragraph{Group chats.}
Have a question that you need to be answered quickly?
Would you like to meet with other squeakers around the world?
A great place to participate in longer-term discussions is the Slack instance at \url{squeak.slack.com}.
Stop by and say ``Hi!''

\paragraph{Other sites.} There are several websites supporting the \sq community today in various ways.
Here are some of them:
\begin{itemize}
  \item \url{github.com/squeak-smalltalk} is the GitHub organization hosting new releases and the various \sq websites.

  \item \url{github.com/OpenSmalltalk/opensmalltalk-vm} is the repository of the OpenSmalltalk-VM which is the virtual machine running \sq.

  \item \url{planet.squeak.org} is the site of PlanetSqueak which is an RSS aggregator.
  It is a good place to get a flood of squeaky things.
  This includes the latest blog entries from developers and others who have an interest in \sq.
\end{itemize}

%=================================================================
\section*{Examples and exercises}

We make use of two special conventions in this book.

We have tried to provide as many examples as possible.
In particular, there are many examples that show a fragment of code that can be evaluated.
We use the symbol \ct{-->} to indicate the result that you obtain when you select an expression and \menu{print it}:

\begin{code}{@TEST}
3 + 4 --> 7    "if you select 3+4 and 'print it', you will see 7"
\end{code}

% ct TODO DISCUSS: Why is this disabled?
% In case you want to play in \sq with these code snippets, you can download a plain text file with all the example code from the book's web site: \sbe.

The other convention that we use is to display the icon \dothisicon{} to indicate when there is something for you to do:

\dothis{Go ahead and read the next chapter!}

%=================================================================
\section*{Acknowledgments (2009 edition)}

% We would like to thank various people who have contributed to this book.
% In particular, we thank
We would like to thank Hilaire Fernandes and Serge Stinckwich who allowed us to translate parts of their columns on \st, and Damien Cassou for contributing the chapter on streams.
We also thank Tim Rowledge for the use of the \sq logo, and Frederica Nierstrasz for the original cover art.

We especially thank Lukas Renggli and Orla Greevy for their comments on drafts of the first release.

We thank the University of Bern, Switzerland, for graciously supporting this open-source project and for hosting the web site of this book.

We also thank the Squeak community for their enthusiastic support of this project, and for informing us of the errors found in the first edition of this book.
Finally, we thank the team that developed Squeak in the first place for making this amazing development environment available to us.

%=============================================================
\ifx\wholebook\relax\else
   \bibliographystyle{jurabib}
   \nobibliography{scg}
   \end{document}
\fi
%=============================================================
